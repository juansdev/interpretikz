% CREACION DE COMANDOS
\newcommand{\antebrazo}[7][1]{
\node[xscale=#1,yscale=#6,rotate=#4] (#2) at (#3) {
    \begin{tikzpicture}
        \shade[left color=#7!70!black,right color=#7!30!white,xshift=1.5cm,yshift=-1.5cm,line width = 0pt,rounded corners=1ex] 
        (1,0)--(1.5,0)--(1.5,0.33)--(1.0,0.30)--(0,0.45)--(0,0)--(1,0);    
    \end{tikzpicture}
    };
}
% \x es el #5
% \c es el #7
\newcommand{\brazo}[7][1]{
\node[xscale=#1,yscale=#6,rotate=#4] (#2) at (#3) {
    \begin{tikzpicture}
        \shade[right color=#7!70!black,left color=#7!30!white,xshift=-0.5cm,yshift=1.5cm,line width = 0pt,rounded corners=1ex] (1.6,0)--(2.,0)--(2.,0.4)--(1.8,0.35)--(1.1,0.60-#5 * 0.01)--(0.8,0.45)--(0.4,0.7)--(0.1,0.3)--(0.3,0)--(1,-0.1 + #5 * 0.005)--(1.6,0);
    \end{tikzpicture}
    };
}
\newcommand{\tronco}[7][1]{
\node[xscale=#1,yscale=#6,rotate=#4] (#2) at (#3) {
    \begin{tikzpicture}
        \shade[top color=#7!30!white,bottom color=#7!70!black, line width = 1pt,rounded corners=2ex,yshift=-0.3cm,xshift=0.2cm] (0.,1)--(0.3,2.16)--(1.3,2.1)--(0.8,0.9)--(0.7,0.1)--(0.5,0.1)--(0.,0.1)--(0.,1);
    \end{tikzpicture}
    };
}
\newcommand{\mano}[7][1]{
    \node[xscale=#1,yscale=#6,rotate=#4] (#2) at (#3) {
        \begin{tikzpicture}
            \shade[top color=#7!30!white,bottom color=#7!70!black, line width = 1pt,rounded corners=2ex,xshift=0cm,yshift=0cm]
                (-0.3,-1)--(-0.3,0.6)--(0,1)--(0.3,0.6)--(0.3,-1)--(-0.3,-1);
            \shade[top color=#7!30!white,bottom color=#7!70!black, line width = 1pt,rounded corners=2ex,xshift=0.6cm,yshift=0cm]
                (-0.3,-1)--(-0.3,0.6)--(0,1)--(0.3,0.6)--(0.3,-1)--(-0.3,-1);
            \shade[top color=#7!30!white,bottom color=#7!70!black, line width = 1pt,rounded corners=2ex,xshift=1.2cm,yshift=0cm]
                (-0.3,-1)--(-0.3,0.6)--(0,1)--(0.3,0.6)--(0.3,-1)--(-0.3,-1);
            \shade[top color=#7!30!white,bottom color=#7!70!black, line width = 1pt,rounded corners=2ex,xshift=1.8cm,yshift=0cm]
                (-0.3,-1)--(-0.3,0.6)--(0,1)--(0.3,0.6)--(0.3,-1)--(-0.3,-1);
                
            \shade[top color=#7!30!white,bottom color=#7!70!black, line width = 1pt,rounded corners=3ex,xshift=-1cm,yshift=0cm]
                (0.0,-0.8)--(-0.7,-0.6)--(-0.1,0.3)--(0.5,0.3)--(0.2,-0.8);
        \end{tikzpicture}
    };
}

% DEFINECOLOR
\definecolor{colone}{RGB}{204,222,210}
\definecolor{coltwo}{RGB}{194,207,189}
\definecolor{colthree}{RGB}{183,192,168}
\definecolor{colfour}{RGB}{173,176,147}
\definecolor{colfive}{RGB}{163,161,126}
\definecolor{colsix}{RGB}{153,147,105}
\definecolor{colseven}{RGB}{142,131,84}
\definecolor{coleight}{RGB}{132,116,63}
\definecolor{colnine}{RGB}{122,101,42}
\definecolor{colten}{RGB}{112,86,21}
\definecolor{coleleven}{RGB}{102,51,0}
\definecolor{colB}{RGB}{243,235,179}
\definecolor{mygray}{RGB}{208,208,208}

\begin{document}
    % \x -> NUMERICO, sus valores son desde el valor 0 hasta coleven (Es decir 11, ya que es el total de variables a recorrer por el bucle).
    % \c -> VALOR RGB.

    % Puede ser \c o \i\j o [count=\x from 0]
    \foreach \c [count=\x from 0] in {colone,coltwo,colthree,colfour,colfive,colsix,colseven,coleight,colnine,colten,coleleven} {
        \begin{frame}
            \begin{center}
                \begin{tikzpicture}[scale =1, node distance=0.6cm,every node/.style={scale=2}]
                    \fill[black](-6,-3)--(-6,5)--(6,5)--(6,-3)--(-6,-3);
                    \fill[black](-5,-2.5)--(-5,-1.6)--(5,-1.6)--(5,-2.5)--(-5,-2.5);
                    \brazo[-1]{M}{2.6,0.3}{-45}{\x}{0.95}{\c}
                    \antebrazo[-1+ \x*0.25/10]{M}{0.86,0.05-\x*0.95/10}{45-\x*45/10}{0}{1-\x*0.2/10}{\c}
                    \tronco[-1]{M}{3.5,-0.05}{0}{0}{1}{\c}
                    \brazo[-0.85]{M}{4,0.3}{-120}{6}{0.9}{\c}
                    \antebrazo[-0.9]{M}{3.7,-0.75}{-5}{0}{0.9}{\c}
                    \shade[right color=\c!70!black,left color=\c!30!white,] (2 + \x *0.01,2.
                    7 + \x *0.01) circle(0.9cm);
                    \mano[-0.12+\x*0.01/10]{M}{-0.25+\x*0.05/10,0.8-\x*1.42/10}{45-\x*5}{0}{-0.12+\x*0.05/10}{\c}
                    \fill[white](-3.2,-2)--(-3.2,-1.25)--(3.2,-1.25)--(3.2,-2)--(-3.2,-2);
                    \brazo[0.85]{M}{-4,0.3}{-120}{6}{0.9}{colB}
                    \antebrazo[0.9]{M}{-3.7,-0.75}{-5}{0}{0.9}{colB}
                    \tronco[1]{M}{-3.5,-0.05}{0}{0}{1}{colB}
                    \brazo[1]{M}{-2.6,0.3}{-45}{\x}{0.95}{colB}
                    \antebrazo[1-\x*0.28/10]{M}{-0.86,0.05-\x*0.85/10}{45-\x*40/10}{0}{1-\x*0.15/10}{colB}
                    \shade[left color=colB!70!black,right color=colB!30!white,] (-2 + \x *0.01,2.7 - \x *0.01) circle(0.9cm);
                \end{tikzpicture}
            \end{center}
        \end{frame}   
    }
\end{document}
